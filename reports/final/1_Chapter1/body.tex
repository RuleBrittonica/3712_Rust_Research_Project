\chapter[Chapter 1]{Literature Review}
\label{chap:lit_review}

\section{Introduction}
\label{sec:lit_intro}
Rust is a modern systems programming language that enforces memory safety
through a strict ownership and borrowing discipline. This discipline, enforced
by Rust's borrow checker, ensures that well-typed Rust programs are free from
data races and dangling pointers without requiring a GC
\cite{automated_refactoring_of_rust_programs}. However, Rust's unique type
system — centered on ownership and lifetimes — poses
new challenges for code transformation tools and formal reasoning. This
literature review surveys academic work on automated refactoring techniques for
Rust, approaches to verification of Rust programs (including formal methods and
model checking), and efforts to formalize Rust's semantics. We compare Rust's
refactoring and verification tooling to those of other languages, and we discuss
integration with development environments. The goal is to highlight the state of
the art in making Rust programs easier to evolve and prove correct, while
pinpointing structural refinements for a clearer organization of this body of
work.

\section{Overview of Refactoring}

As previously mentioned, refactoring is the process of restructuring existing
code to improve its internal structure wouthout changing its external behavior.
It is a key practice for enhancing code maintainability and reducing technicanle
debt in software projects \cite{OneThousandOneStories-SoftwareRefactoring}.
Modern IDES, such as Eclipse and IntelliJ IDEA include a wide array fo automated
refactorings, providing developers with quick, semantics- perserviding code
transformations \cite{AdventureOfALifetime}. In theory, these automated
refactorings offer a safe way to restructure code (backed by well-defined
preconditions to preserve behaviour \cite{Formal_Specifiation_JAVA}) and should
be widely used in practice.

However, empirical studies show a disparity between tool support and actual
usage. Developers often remain reluctant to use automated refactoring features,
preferring manual code changes
\cite{OneThousandOneStories-SoftwareRefactoring}. For example, even with
“Rename” refactoring available at the click of a button, many programmers still
rely on find-and-replace or other manual methods
\cite{OneThousandOneStories-SoftwareRefactoring}. The survey also noted that
many developers expressed a desire to better understand their IDE's refactoring
capabilities. This suggests that while the tools exist, there is a gap in
awareness and useability of the tooling available. Additionally, as evidenced by
the lack of adoption of even ``Rename'' refactorings,  challenges still remain
in getting developers to trust automated tools to perform complex manipulations.

Research into automated refactoring aims to bridge this gap by making tools more
reliable and intelligent. For instance, Tip (2007) demonstrated that
refactorings can be modeled with type constraints to systematically explore
alternative valid program structures \cite{RefactoringUsingTypeConstraints}.
Such approaches use static analyses to ensure that transformations maintain type
correctness and behavioral equivalence, giving developers confidence in the
automation. Nevertheless, the human factor (usability, understandability of tool
behavior) remains critical.
\section{Rust's Ownership Model and Formal Semantics}
\label{sec:rust_owndership_model_formal_semantics}

Rust's ownership model introduces compile-time enforced rules for aliasing and
memory lifetime. Each value in Rust has a single owning scope, and references
(borrows) must obey strict rules that prevent concurrent mutation and
use-after-free \cite{automated_refactoring_of_rust_programs},\cite{the_rust_language}.
While this model provides strong safety guarantees, it complicates formal
semantics and tool development. Early efforts to rigorously define Rust's
semantics culminated in \textit{RustBelt}, which provided the first machine-checked
safety proof for a realistic subset of Rust's type system
\cite{RustBelt}.\textit{RustBelt} demonstrated that safe Rust code (even when
using unsafe internals of standard libraries) is memory and thread safe, by
modeling Rust in the Coq proof assistant \cite{RustBelt}. This foundational work
established confidence in Rust's core design and set the stage for verifying
more complex properties. Subsequent research has extended Rust's semantic
foundation to cover aliasing in unsafe code (e.g. Stacked Borrows and Tree
Borrows models for pointer aliasing) \cite{AENEAS_PART_2}, and to connect Rust's
high-level rules with low-level memory models. These formal models provide a
basis on which verification tools and refactoring tools can reason about Rust
code behavior. In summary, Rust's enforced aliasing discipline, while posing
challenges, has inspired a rich line of work in formal semantics that underpins
safe refactoring and verification.

\section{Automated Refactoring in Rust vs Other Languages}
\label{sec:automated_ref_rust_vs_other}
Welll established languages like java and C\# benefit from decades of IDE
support for refactorings such as \textit{Rename}, \textit{Extract Method} and
\textit{Extract Module / Variable}.
Rust, being newer and with far more complex compile-time rules, has lagged
behind in automated refactoring support \cite{AdventureOfALifetime}. The
ubiquitous \textit{Extract Method} is ``widely used in all major IDEs'' for
other languages, but implementing it for Rust is surprisingly non-trivial due to
Rust's ownership and lifetime constraints \cite{AdventureOfALifetime}. In
contrast to a language like Java, where extracting a function involves mostly
syntactic rearrangement, Rust's refactoring tools must also infer where to
borrow or clone data and how to introduce lifetime parameters to satisfy the
borrow checker \cite{automated_refactoring_of_rust_programs},
\cite{automatically_enforcing_rust_trait_properties}. Early comparisons noted
that Rust's lack of reflection and unstable compiler APIs made it harder to
build refactoring tools as robust as those for Java / C\#. Even in the two years
since REM was released, compiler APIs have changed so much that it proved
impossible to compile the original project without significant rewrites. On the
positive side, Rust's compiler errors can guide manual refactorign by
pinpointing violations of borrowing rules, giving programmers immediate
feedback; this meansd that if a refactoring compiles in Rust, it is likely
behaviour-preserving \cite{Endler_2024}. Nonetheless, the consensus in both
academia and industry is that first-class refactoring tools are needed to manage
large Rust codebases with the same ease developers expect in other environments
\cite{AdventureOfALifetime}, \cite{OneThousandOneStories-SoftwareRefactoring}.
\section{Automated Refactoring Techniques for Rust}
\label{sec:automated_ref_tech_rust}
\subsection{Renaming and Simple Refactorings}
One of the first efforts to build a Rust refactoring tool was by G. Sam et al.
(2017), who created a proof-of-concept refactoring framework utilizing the Rust
compiler's internal APIs. The team partnered with Mozilla Research to be among
the first to implement the Rust specific refactorings of \textit{Lifetime
Elision} and \textit{Lifetime Reification}. This allowed their program to
introuce explicit lifetime parameters in instances where the compiler was
implicitly inferring them - a refactoring that was brand new to Rust. The
challenges encountered illustrated how Rust's stricter scoping and shadowing
rules required careful handling of name conflicts during renaming (e.g. avoiding
situations where renaming a variable could unintentionally shadow another)
\cite{automated_refactoring_of_rust_programs}. Additionally, the 2017 study
concluded that many refactorings are possible with Rust's compiler
infrastructure, but ensuring \textit{behavioural preservation} (especially
around ownership transfers) requires additional static analyses not needed in
langauges without Rust's constraints. Their work on \textit{Lifetime Elision}
has since formed part of the REM toolchain, where the rich feedback from the
Rust compiler is leveraged to ensure that the final transformation is both
valid Rust and as legible as possible \cite{AdventureOfALifetime},
\cite{BorrowingWithoutSorrowing}.

\subsection{Extract Method (REM)}

A major advance in Rust refactoring came with the Rust Extract Maestro (REM),
with the theoretical background provided by the work of Costea et al. (2023) in
\textit{Adventure of a Lifetime: Extract Method Refactoring for Rust}. Thy et
al. (2023) then provided a practical implementation of REM in \textit{Borrowing
Without Sorrowing}. REM is a tool that tackled notoriously complex Extract
Method refactoring for Rust. REM decomposes the extract-method process into a
sequence of transformations, each addressing a specific aspect of Rust's type
and borrow rules \cite{AdventureOfALifetime}. The approach begins with a
``na\"ive'' function hoisting (moving a block of code into a new function), then
applies a series of \textit{automated program repairs} to make the code compile
correctly. This process is guided by \textit{oracles} - specialised analyses
that resolve issues the compiler would flag. One such oracle is an
intra-procedural \textbf{ownership analysis} which infers wheter each value
moved to the new function should be passed by value, by shared reference (using
\texttt{\&T}), or by mutable reference (using \texttt{\&mut T}). Appendix 1
(\ref{sec:appendix1}) provides a detailed example of what this looks like in
practice.

Another oracle leverages Rust's own compiler (rustc) as a sub-procedure: REM
invokes Rust's lifetime checker to identify lifetime inconsistencies, then
introduces appropriate lifetime parameters to the extracted function until the
borrow checker is satisfied.  By iteratively repairing borrow errors in the
extracted code, REM ensures the transformation preserves semantics and yields a
well-typed program. The REM tool, implemented as an extension to the IntelliJ
Rust plugin, was shown to handle complex extractions that involve non-local
control flow (e.g. \texttt{return} or \texttt{break} inside the extracted fragment) and borrowing
across function boundaries. In an evaluation on real-world Rust projects, REM
could successfully perform extractions that developers had done manually
(including cases with nested lifetimes), outperforming existing tools in the
scope of code it can handle. Their work demonstrated that seemingly ``unsafe''
transformations (like cutting a function in half) can be automated in Rust by
coupling transformations with verification from the Rust compiler's own checks.

\subsection{Automated Fixes for Ownership Errors}
An alternative angle on Rust refactoring is automatically \textit{fixing} code that
violates ownership/borrow rules - essentially a refactoring from
``non-compiling'' to ``compiling'' code without changing behavior.
\textbf{Rust-Lancet} (Yang et al., ICSE 2024) is a tool aimed at
\textit{automated ownsership-rule-violation fixing with behaviour preservation}.
Given a Rust source file that fails to compile due to borrow checker errors,
Rust-Lancet analyzes the Abstract Syntax Tree (AST) and applies code transformations
to eliminate the error while preserving the program's semantics
\cite{RustLancet}. Under the hood, it uses Rust parsing libraries (\texttt{syn} and
\texttt{quote}) to manipulate the AST and generate patched code. In a similar
manner to REM, Rust-Lancet employs an interative repair loop: it first applies a
fix (for example, inserting a \texttt{clone()} on a value to appease ownership
rules), then rechecks the compiler errors and repeats if futher fixes are
needed. The tool includes correctness checks to avoid over-fixing; a
\textit{behaviour preservation} module validates that the inserted changes do
not alter program outputs for typical cases.  In an evaluation on over 150 real
cases of ownership rule violations (gathered from Rust's test suite and prior
research), Rust-Lancet was able to completely fix a large fraction of them with
zero false positives (i.e. it never introduced an incorrect fix). It even
outperformed suggestions given by the Rust compiler and baseline techniques
using large language models, especially in scenarios requiring multiple
coordinated edits. While Rust-Lancet is positioned as a bug-fixing tool, it
overlaps with refactoring by automatically rewriting code in a
semantics-preserving way. Its success highlights how Rust's rigorous rules can
be leveraged - the tool knows a patch is correct when the compiler's checks pass
and its behavior preservation validation succeeds \cite{RustLancet}.

\subsection{Comparison and Summary}
Overall, automated refactoring in Rust is characterized by taking full advantage
of the rich and sophisticated output of the Rust compiler. Tools like REM
explicitly invoke the borrow checker and type checker as so called oracles
\cite{AdventureOfALifetime}, whilst others, like Rust-Lancet, effectively
automate a ``fix until compile'' loop wiht guarantees of behaviour preservation.
This is in contrast to many refactoring tools for garbage-collected languages,
which operate mostly on the AST or intermediate representation without needing
to consider memory lifetimes. The research so far indicates that, with suitable
abstractions (e.g. ownership analysis algorithms and lifetime inference
strategies), even complex Rust refactorings can be automated soundly. As more of
these techniques mature, we expect Rust's refactoring support to begin matching
the breadth available for languages like C++ and Java, albeit with different
underlying algorithms tailored to Rust's unique semantics.

\section{Verification and Formal Methods for Rust}
\label{sec:verification_formal_methods}
Rust's appeal for building reliable software has spurred interest in applying
formal verification techniques to Rust programs. Traditional \textit{deductive
verification} and \textit{model checking} must be adapted to account for Rust's ownership,
lifetimes, and possibly unsafe code. In this section, we review key verification
tools and techniques, and the formal semantics advances enabling them.

\subsection{RustBelt and Type-System Soundness}

Jung et al. (2018) introduced \textbf{RustBelt}, a foundational framework for
reasoning about Rust's type system and memory safety. In their landmark paper,
\textit{RustBelt: Securing the Foundations of the Rust Programming Language},
they proved the soundness of Rust's core type system. By embedding a Rust-like
language ($\lambda$Rust) in the Iris framework (a higher-order concurrent
separation logic), RustBelt verified that well-typed safe Rust code cannot
exhibit undefined behaviour \cite{RustBelt}. This was the first independent verification of
Rust's claims about memory and thread safety. Additionally, it proivdfed a
foundation on which other verification effors could build: one can assume that
safe code is memory-safe, allowing them to focus verification effort either on
higher-level functional correctness or the remaining pitfalls of unsafe code.

\subsection{Verification via Functional Translation (CHARON and AENEAS)}
One prominent approach to verifying Rust programs is to reduce the problem to
verifying \textit{pure functional code}. \textbf{AENEAS} (Ho \& Potzenko, ICFP
2022) is a toolchain that translates a substantial subset of Rust into a purely
functional language, suitable for input to existing proof assistants or model
checkers \cite{AENEAS}. The key insight of Aeneas is that for many Rust programs
(those not using interior mutability or unrestricted unsafe code), explicit
memory reasoning can be eliminated. Aeneas introduces an intermediate
representation called \textit{Low-Level Borrow Calculus} (LLBC), inspired by
Rust's mid-level IR (MIR), which makes the ownership and borrowing structure of
a program explicit. Crucially, LLBC's semantics is \textit{ownership-centric}:
instead of a heap, it uses an abstraction of \textit{loans} and \textit{borrows}
to track aliasing, meaning individual values carry information about what they
borrow.

Ho and Protzenko went on to define a set of pure, value-based semantics for LLBC
- one with no pointer addresses - therby ``capturing the essence of the borrow
mechanism'' in a mathematical form \cite{AENEAS_PART_2}. They then translate LLBC
into a pure lambda calculus\footnote{See this excellent introduction to the
Lambda Calculus for more information: \newline
https://plato.stanford.edu/entries/lambda-calculus/}, effectively converting
Rust code into a functional
program with the same behavior. The result is that one can verify properties of
the Rust program by verifying the translated functional program using a theorem
prover of choice. For example, one might translate Rust to Coq or Fstar and
prove postconditions about the output, with Aeneas ensuring those properties
carry back to the original Rust code.

To tackle tricky issues like termination of borrows across function boundaries,
AENEAS introduced a notion of \textit{backwards functions} to conservitavely
approximate the ``borrow graph'' of a program and ensure the translated code
remains termination-friendly. The approach was shown to significantly reduce the
burden on proof engineers, since they no longer needed to reason about low-level
memory safety - Rust's type system has already done the heavy lifting for them.

From there, Ho et al., (ICFP 2024) presented \textit{Sound Borrow-Checking for
Rust via Symbolic Semantics} \cite{AENEAS_PART_2}, with the goal of addressing two
open questions from AENEAS's methodology.First, they established a link between
Aeneas's LLBC and a traditional heap-based semantics, by proving that LLBC's
admittedly unusual modelling is faithful to a standard memory model, ensuring
that verifiying programs in LLBC is not ``proving the wrong thing''. Second,
they introduced a set of formal \textit{symbolic semantics} for Rust that act as
a sound abstraction of concrete execution, and prove that these symbolic
semantics are capable of correctly approximating Rust's behavior.  In essence,
they show that if a program is accepted by their symbolic interpreter (which
enforces borrow rules), then there exists a corresponding concrete execution in
a real heap that is memory-safe. This result formally validates the idea that
the borrow checker (or a tool mimicking it) prevents all illicit behaviors.

A very important outcome of this work is a proof that the \textbf{symbolic
interpreter can server as a verified borrow-checker} for LLBC programs. This
provides increased confidence that tools like Aeneas (which rely on symbolic
execution as an intermediate) are built on solid theoretical ground, and it
paves the way for future verified compilers or static analyzers for Rust that
incorporate a proven-correct borrow checker. The broader significance is that
Rust's complex lifetime rules are now accompanied by a machine-checked proof of
their soundness, something rare for industrial language features.

\subsection{Static Analysers and Model Checkers}
\subsubsection*{Static Analysis}
Beyond full formal verification, several tools bring lightweight verification or
bug-finding to Rust. \textbf{PRUSTI} (Astrauskas et al., OOPSLA 2019) is a
static \textit{deductive verifyer} that uses Rust's types to simplify program
annotation and verification, \cite{prust_in_practice}.
Prusti translates Rust code into the Viper verification language, using Rust's
lifetime and ownership information to implicitly encode permissions for
verification \cite{prusti_project}. This means a developer can write
specifications (pre/post-conditions, invariants) for Rust functions, and Prusti
will prove them, assuming no undefined behavior in safe code. The advantage of
leveraging Rust's type system is evident - e.g., Prusti knows that two mutable
references cannot alias, which corresponds to a simple permission separation in
the verification condition. Over the past few years, Prusti has grown to handle
significant subsets of Rust, and it even provides a Visual Studio Code extension
for interactive use \cite{prusti_github}.

In a similar vein, \textit{RefinedRust: A Type Sysem for High-Assurance
Verification of Rust Programs} (G\"aher et al., 2024) introduced a
refinement type system for Rust that can verify both safe and certain unsafe
code by embedding checking into the type system \cite{RefinedRust}.
RefinedRust's types allow expressing rich safety and correctness properties, and
a prototype tool translates annotated Rust into a Coq model for verification.
This line of work brings Rust closer to languages like Dafny or Liquid Haskell
in terms of having a path to verification integrated with the language's types.

\subsubsection*{Model Checking}
The most notable tool for \textit{model checking} is \textbf{Kani}, a bounded
model checker for Rust developed by AWS. Kani translates Rust programs
(including those with \texttt{unsafe} code) into verification conditions that are checked
by a SAT/SMT solver (built on the C Bounded Model Checker (CBMC) backend)
\cite{automatically_enforcing_rust_trait_properties}. What sets Kani apart is
that it operates on Rust's MIR (Mid-level IR), leveraging information about
traits and generics directly rather than reducing to LLVM bitcode. By working at
the MIR level, Kani retains high-level semantics like trait object behavior,
which allowed the team to verify complex properties involving dynamic trait
objects (Rust's form of dynamic dispatch). VanHattum et al. (ICSE 2023) describe
how Kani handles dynamic trait dispatch soundly, making it the first model
checker to support the full range of Rust trait features
\cite{verifying_dynamic_trait_objects}. In experiments, they found that using
Rust-specific knowledge (MIR and trait info) improved verification performance
by 5$\times$ - 15$\times$ compared to an equivalent analysis at the LLVM level.
Kani is being used to verify critical systems components (for example, AWS has
applied it to verify parts of their cryptographic libraries and virtualization
code), focusing on memory safety and user-provided assertions in
performance-critical Rust code. Other dynamic analyzers like MIRI (an
interpreter that detects undefined behavior in Rust by executing code in a
defined environment) complement these verification tools by checking one
execution at a time but with a semantic model that catches unsafe violations;
however, MIRI is more a testing aide than a formal verifier.

\subsection{Summary}
We have shown that the Rust verification ecosystem in incredibly rich and includes: foundational proofs of the
language's safety (RustBelt, symbolic semantics), translation-based verifiers
(Aeneas, Prusti, RefinedRust) that build on the guarantees of Rust's types, and
model checkers (Kani) that leverage Rust's own compiler internals for efficient
exhaustive exploration. Each of these harnesses Rust's key feature—its rich type
system—as an asset to simplify or optimize verification. This contrasts with
verification for languages like C/C++, where the lack of a guaranteed memory
safety baseline means much effort is spent just to mitigate wild pointers or
data races. In Rust's case, the baseline guarantees free the verifier to focus
on higher-level properties or the remaining unsafe corners of the language. The
research surveyed here shows a trend of \textit{integrating verification with the
language's design}: from using the borrow checker in a proof assistant, to
encoding Rust lifetimes as logical predicates, to running model checkers on MIR.
This tight integration yields verifiers that are both sound (no false negatives
on safety) and relatively precise, making formal assurance of Rust software
increasingly attainable.

\section{Integration with IDEs and Language Servers}
\label{sec:integration_language_server}
The ultimate goal for refactoring and verification tools is seamless integration
into developers' workflows. For Rust, integration has meant working closely with
the compiler or the emerging Language Server Protocol (LSP). However,
\texttt{rustc} was not originally desinged as a reusable library for IDE
features, which led to the creation of separate projects to support code
intelligence and analysis. The first generation Rust Language Server (RLS)
attempted to use the compiler internals to power IDE features, but it struggled
with performance and completeness as Rust evolved. Today, the community relies
on \textbf{rust-analyzer}, a from-scratch implementation of a Rust language
server that emphasizes speed and modularity \cite{Schiedt_2022}. Rust-analyzer
constructs its own parse tree and does its own type inference, enabling
on-the-fly responses to editor queries without running the full compiler on
every change. This architecture is conducive to implementing refactorings as
lightweight code transformations called “assists.” In fact, rust-analyzer
already supports basic refactorings like \textit{Rename} and \textit{Extract
Variable}, and \textit{Extract Method} as editor assists. These operate on
rust-analyzer's internal syntax tree and use its knowledge of ownership and
types to ensure the edits are valid. For example, the Extract Function assist
initially had limitations with Rust's generics and lifetimes - extracting code
from a generic function would produce errors because the assist didn't propagate
required type parameters or lifetime parameters to the new function. However,
even with substantial development by the community, certain assists still fall
significantly short of the capabilities of tools developed specifically for
purpose, such as REM.

Integration into the popular IntelliJ IDEA (via the Rust plugin) has also seen
some development. The REM tool was originally built atop the IntelliJ plugin,
indicating that industrial IDEs can serve as a proving ground for research
prototypes \cite{AdventureOfALifetime}. In principle, once validated, these
prototypes can be upstreamed, bringing advanced refactorings to everyday
developers. A challenge here is maintaining the tools: Rust's syntax and
compiler behavior evolve, so refactoring algorithms must be kept in sync. One
way this is being addressed is by using Rust's stable analysis APIs where
possible. For instance, tools like Rust-Lancet chose to use the
community-maintained \texttt{syn} crate for parsing so that they aren't tied to
intermal compiler data structures that change nightly \cite{RustLancet}.
Similarly, projects to support large-scale refactoring use cases (such as
Facebook's code migration tools or the Rust-to-Rust transformations needed in
the Rust 2018 edition uplift) relied on either compiler suggestions or external
scripting. The Rust compiler team has introduced machine-applicable suggestions
in compiler error messages, which tools like \textbf{rustfix} can apply
automatically. This essentially allows the compiler to assist in simple
refactors (e.g. renaming a deprecated syntax, adding a missing lifetime
specifier) by emitting a structured suggestion. Whist not as general as a true
refactoring engine, this has helped with batch changes across large codebases.

\subsection{Compiler Internals for Ownership and Borrowing}
It's worth noting how Rust's compiler architecture influences refactoring and
verification tools. Rustc itself goes through a series of analyses: it parses
into an AST (with desugaring), performs name resolution and type checking
(including lifetime inference), and then produces MIR where
borrow checking and other analyses occur. Because of this, tools integrated at various stages
have different views of the program. For example, a refactoring tool working on
an AST (like rust-analyzer's assists or Rust-Lancet) must re-run portions of
the compiler (or mimic them) to ensure that after transformation the code still
passes borrow check. Meanwhile, a verification tool like Kani that operates on
MIR has the advantage of piggybacking on the actual borrow-checker's results -
it can trust that MIR is already free of borrow rule violations and focus on
exploring execution paths \cite{verifying_dynamic_trait_objects}. As research
progresses, we may see more refactoring tools move to operate on MIR or use the
borrow checker in the loop, because MIR provides a simpler, desugared view of
the code with explicit borrow regions (e.g. the Polonius project, a
next-generation borrow checker, could potentially expose an API for tools to
query borrow relationships). For now, practical refactoring uses higher-level
representations (AST plus perhaps some compiler query for type info) because
it's easier to map edits back to source code.

\section{Related Work Beyond Rust: Refactoring in Other Languages and Paradigms}
\label{sec:related_work_beyond_rust}
The challenges and advances in refactoring outlined throughout this review are
not unique to Rust. Many other languages have ispired research and into
automated refactoring and its many pitfalls. The easiest case study is Java,
being one of the earliest targets of refactoring tools, and it has a rich
literature. Beyond the generics case study mentioned earlier
(\cite{GenericRefactoringJAVA}), researchers have looked at refactoring
concurrent Java code, refactoring to use new language features, and large-scale
restructuring of legacy systems. Zhang et al. (2024) recently proposed an
automated refactoring approach for Java's asynchronous programming using the
CompletableFuture API. Their tool, \textbf{ReFuture}, integrates static analysis
(visitor patterns, alias analysis, etc.) to identify where an older async
construct (like using raw threads or Futures) can be transformed into a modern
CompletableFuture chain \cite{AutomaticRefactoringAsyncJAVA}. Impressively,
\textbf{ReFuture} was evaluated on 9 large Java projects, including
\textit{Hadoop} and \textit{ActiveMQ}, and managed to refactor 639 out of 813
eligible code segments automatically, without introducing errors
\cite{AutomaticRefactoringAsyncJAVA}. This demonstrates that automated
refactoring can manage to tackle performance or paradigm-migration tasks, not just
cosmetic and maintenance changes.

In the mobile app domain, Lin et al. (2015) addressed refactoring for Android's
asynchronous constructs. Android apps often mis-use the AsyncTask class, leading
to memory leaks or lost UI updates. Lin and his team developed
\textbf{ASYNCDROID}, a tool to refactor improper AsyncTask usage into a sover
structures like IntentService \cite{AndroidAsncRefactoring}. Their formative
study found that about 45\% of the AsyncTask occurrences in real apps could be
automatically refactored by ASYNCDROID's rules, and an additional ~10\% could
be handled with minor manual tweaks, resulting in more than half of problematic
cases being fixed largely automatically. The tool was implemented as an Eclipse
plugin and offered typical refactoring tool conveniences including a preview of
the changes and a rollback option. Importantly, when the authors submitted some
of these refactoring patches to open-source projects, the maintainers accepted
them, validating that the transformations were indeed considered improvements
with no behavioral regressions \cite{AndroidAsncRefactoring}. This work
illustrates how automated refactoring can be applied in a specialized context
(mobile asynchronous UI code) by encoding domain-specific knowledge as
transformation rules and preconditions.

Other notable refactoring research outside of Rust includes work on functional
languages and dynamically-typed languages. In Haskell, for example, the HaRe
tool (Li et al., 2005) brought refactoring support to a lazy, purely functional
language, requiring careful handling of Haskell's scoping, type classes, and
purity concerns \cite{HaRe}. In dynamic languages like JavaScript and Python, refactoring is
complicated by the lack of static types; researchers have explored heuristic and
runtime-analysis-based approaches to perform refactorings safely (e.g.,
refactoring Python's modules or JavaScript's callbacks to promises). While we
will refrain from delving into specific studies for those here, it is worth noting that each
language tends to spawn its own refactoring research to address unique features.

Finally, it is important that we note that refacotring is very closely
intertwined with other software engineering tasks like code smell detection,
program repair and modernisation. Refactoring tools are sometimes used as
building blocks for automated program repair, such as applying
behavior-preserving transformations to enable a later bug-fixing change, with
one such case even mentioned in
\textit{Adventure of a Lifetime} \cite{AdventureOfALifetime}. Tools like
Facebook's JScodeshift or Python's Bowler allow developers to script custom
refactorings (essentially semi-automated transformations), which blurs the line
between a ``human-driven'' refactoring ands a tool-driven migration. The
incdeasing adoption of language servers means refactoring capablilities can be
made available in a wide range of editors, not just heavyweight IDEs
\cite{AdventureOfALifetime}. This has been a positive development for languages
like Rust and Go, where a language server can provide refactoring assists
consistently across many development environments.

\section{Conclusion} % TODO expand the conclusion, potentialy with a section
% going over the gaps in the research. Alternatively, we could analyse each
% source at the time it is introduced and discuss its limitations there.
\label{sec:lit_concusions}
The landscape of Rust refactoring and verification has rapidly evolved, marrying
the Rust compiler’s strengths with novel algorithms from the research community.
Automated refactoring tools for Rust have progressed from simple renaming
utilities to sophisticated systems that integrate program repair and static
analysis (REM and Rust-Lancet) to handle ownership and lifetime challenges
\cite{AdventureOfALifetime}, \cite{RustLancet}. Verification tools leverage
Rust’s unique guarantees to simplify proofs – whether by translating to pure
functional models (Aeneas) \cite{AENEAS}, by encding lifetime reasoning into
logical frameworks (RustBelt, RefinedRust) \cite{RustBelt}, \cite{RefinedRust},
or by exploring executions with model checkers (Kani) that understand
Rust-specific features \cite{verifying_dynamic_trait_objects}. A unifying theme
is that Rust’s ownership semantics, once seen purely as a hurdle, have become an
enabler for both refactoring and verification: they localize the reasoning about
memory, allowing tools to either confidently transform code or assert properties
with the guarantee that certain classes of bugs are already ruled out.

In conclusion, Rust’s combination of strong static checks and low-level
performance has fostered a unique blend of research. The advances in refactoring
mean that developers can expect increasing assistance in restructuring Rust code
without fear of breaking it. Meanwhile, the maturation of formal methods for
Rust suggests that full verification of critical Rust code is becoming
practical, supported by tools that integrate with Rust’s development ecosystem.
By structuring the literature along clear axes (refactoring vs. verification,
formal foundations vs. practical tools, compiler phases involved, safe vs.
unsafe scope), future reviews can further improve clarity and coverage. The
works cited here collectively show that Rust’s initially steep learning curve
for both humans and tools can be mitigated by clever automation and rigorous
semantics – turning Rust’s famed safety guarantees from a burden into a powerful
ally for program analysis and transformation.

\renewcommand\thefigure{\thechapter .\arabic{figure}}