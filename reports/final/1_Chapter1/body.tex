\chapter[Chapter 1]{Literature Review}
\label{chap:lit_review}

\section{Introduction}
\label{sec:lit_intro}
Rust is a modern systems programming language that enforces memory safety
through a strict ownership and borrowing discipline. This discipline, enforced
by Rust's borrow checker, ensures that well-typed Rust programs are free from
data races and dangling pointers without requiring a GC
\cite{automated_refactoring_of_rust_programs}. However, Rust's unique type
system — centered on ownership and lifetimes — poses
new challenges for code transformation tools and formal reasoning. This
literature review surveys academic work on automated refactoring techniques for
Rust, approaches to verification of Rust programs (including formal methods and
model checking), and efforts to formalize Rust's semantics. We compare Rust's
refactoring and verification tooling to those of other languages, and we discuss
integration with development environments. The goal is to highlight the state of
the art in making Rust programs easier to evolve and prove correct, while
pinpointing structural refinements for a clearer organization of this body of
work.

\section{Rust's Ownership Model and Formal Semantics}
\label{sec:rust_owndership_model_formal_semantics}

Rust's ownership model introduces compile-time enforced rules for aliasing and
memory lifetime. Each value in Rust has a single owning scope, and references
(borrows) must obey strict rules that prevent concurrent mutation and
use-after-free \cite{automated_refactoring_of_rust_programs},\cite{the_rust_language}.
While this model provides strong safety guarantees, it complicates formal
semantics and tool development. Early efforts to rigorously define Rust's
semantics culminated in \textit{RustBelt}, which provided the first machine-checked
safety proof for a realistic subset of Rust's type system
\cite{RustBelt}.\textit{RustBelt} demonstrated that safe Rust code (even when
using unsafe internals of standard libraries) is memory and thread safe, by
modeling Rust in the Coq proof assistant \cite{RustBelt}. This foundational work
established confidence in Rust's core design and set the stage for verifying
more complex properties. Subsequent research has extended Rust's semantic
foundation to cover aliasing in unsafe code (e.g. Stacked Borrows and Tree
Borrows models for pointer aliasing) \cite{AENEAS_PART_2}, and to connect Rust's
high-level rules with low-level memory models. These formal models provide a
basis on which verification tools and refactoring tools can reason about Rust
code behavior. In summary, Rust's enforced aliasing discipline, while posing
challenges, has inspired a rich line of work in formal semantics that underpins
safe refactoring and verification.

\section{Automated Refactoring in Rust vs Other Languages}
\label{sec:automated_ref_rust_vs_other}
Welll established languages like java and C\# benefit from decades of IDE
support for refactorings such as \textit{Rename} and \textit{Extract Method}.
Rust, being newer and with far more complex compile-time rulse, has lagged
behind in automated refactoring support \cite{AdventureOfALifetime}. The
ubiquitous \textit{Extract Method} is ``widely used in all major IDEs'' for
other languages, but implementing it for Rust is surprisingly non-trivial due to
Rust's ownership and lifetime constraints \cite{AdventureOfALifetime}. In
contrast to a language like Java, where extracting a function involves mostly
syntactic rearrangement, Rust's refacotring tools must also infer where to
borrow or clone data and how to introduce lifetime parameters to satisfy the
borrow checker \cite{automated_refactoring_of_rust_programs},
\cite{automatically_enforcing_rust_trait_properties}. Early comparisons noted
that Rust's lack of reflection and unstable compiler APIs made it harder to
build refactoring tools as robust as those for Java / C\#. Even in the two years
since REM was released, compiler APIs have changed so much that it proved
impossible to compile the original project without significant rewrites. On the
positive side, Rust's compiler errors can guide manual refactorign by
pinpointing violations of borrowing rules, giving programmers immediate
feedback; this meansd that if a refactoring compiles in Rust, it is likely
behaviour-preserving \cite{Endler_2024}. Nonetheless, the consensus in both
academia and industry is that first-class refactoring tools are needed to manage
large Rust codebases with the same ease developers expect in other environments
\cite{AdventureOfALifetime}, \cite{OneThousandOneStories-SoftwareRefactoring}. 
\section{Automated Refactoring Techniques for Rust}
\label{sec:automated_ref_tech_rust}

\section{Verification and Formal Methods for Rust}
\label{sec:verification_formal_methods}

\section{Integration with IDEs and Language Servers}
\label{sec:integration_language_server}

\section{Conclusions and Insights}\
\label{sec:lit_concusions}

\renewcommand\thefigure{\thechapter .\arabic{figure}}