\section{Appendix 1: Passing by Value and Reference}
\label{sec:appendix1}

This appendix provides three short, self-containted examples of ``na\"ive''
function hoisting in Rust. In each example, we will define a function that is
getting refactored, and we will demonstrate the case where the arguments to the
function are passed by value and the case where they are passed by reference. To
achieve this, we are imagining an automated refactoring tool that initially
takes an inline block of code and moves it into a new function. The compilation
errors (if any) guide a subsequent ownership analysis, which deduces how each
variable should be passed: by value (\texttt{T}), by shared reference
(\texttt{\&T}), or by mutable reference (\texttt{\&mut T}).

\subsection{Example 1: Passing by Value}
\subsubsection*{Original Code (inline)}
\inputminted{rust}{6_appendices/app1_code/ex1_oc.rs}

\subsubsection*{Na\"ive Hoisted Code (before fixes)}
Suppose an automated refactoring tool decides to “hoist” the multiplication
logic into a new function: \newline
\inputminted{rust}{6_appendices/app1_code/ex1_nhc.rs}
In this example, no errors occur because after calling
\texttt{hoisted\_block(x)}, we do not need \texttt{x} again in \texttt{main}, so
moving (passing ownership) is safe. The automated repair analysis detects that
\texttt{x} is no longer needed in \texttt{main} and decides it can be passed by
value.

\subsection{Example 2: Passing by Shared Reference}
\subsubsection*{Original Code (inline)}
\inputminted{rust}{6_appendices/app1_code/ex2_oc.rs}

\subsubsection*{Na\"ive Hoisted Code (before fixes)}
Let's say the refactoring tool decides to hoist the usage of \texttt{get\_first\_word} (or
a bit more logic around it) into a new helper function:
\inputminted{rust}{6_appendices/app1_code/ex2_nhc.rs}
This version won't compile as expected if the function
\texttt{print\_first\_word\_block} takes \texttt{text} by value. Once \texttt{text}
is moved into \texttt{print\_first\_word\_block}, we can no longer use
\texttt{text} afterward in \texttt{main}. The compiler will complain that
\texttt{text} is moved, thus invalidating the line \texttt{println!("Total
length is: {}", length);} if we needed \texttt{text} for anything else.

\subsubsection*{Automated Repair Yields}
\inputminted{rust}{6_appendices/app1_code/ex2_ar.rs}
\noindent What's going on?
\begin{itemize}
    \item \textbf{Read-only usage}: The function \texttt{print\_first\_word\_block}
    just needs to read the string (in order to print the first word).
    \item \textbf{Shared reference}: Because the string must remain valid
    afterward in \texttt{main}, the ownership analysis decides that passing by
    shared reference (\texttt{\&String}) is correct.
\end{itemize}

\subsection{Example 3: Passing by Mutable Reference}
\subsubsection*{Original Code (inline)}
\inputminted{rust}{6_appendices/app1_code/ex3_oc.rs}

\subsubsection*{Na\"ive Hoisted Code (before fixes)}
Hoisting the push operation into a new function could look like this:
\inputminted{rust}{6_appendices/app1_code/ex3_nhc.rs}
Here, we have a compilation error similar to Example 2: once \texttt{values} is
passed by value, ownership is transferred, and we cannot use \texttt{values}
again in \texttt{main}. However, in this case, we truly do need to mutate
\texttt{values}, and we want \texttt{main} to see the updated vector.

\subsubsection*{Automated Repair Yields}
\inputminted{rust}{6_appendices/app1_code/ex3_ar.rs}
\noindent What's going on?
\begin{itemize}
    \item \textbf{Mutation required}: The code needs to modify the original
    vector.
    \item \textbf{Mutable reference}: By passing \texttt{\&mut Vec<i32>}, the
    function can mutate \texttt{values} in place, and \texttt{main} retains
    ownership of \texttt{values} to use afterward.
\end{itemize}
