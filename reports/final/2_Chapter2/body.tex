\chapter{Expanding the Capabilities of REM}
\label{chap:expanding_rem}

%  Approx 8-10 pages

% Quick introduction to what this chapter will cover
% Emphasize the first innovation, being how the project has expanded the
% capabiliteis of REM beyond its original scope
% Statement of what this chapter will demonstrate:
% - generic code
% - support for async code
% - standalone CLI that handles all the logic
% - integration with IDEs (VSCode)

Automated refactoring in Rust has long been constrained by the languages strict
ownership, borrwing and lifetime rules. Early prototypes of the Rusty Extraction
Maestro demonstrated that \emph{Extract Method} refactorings could be made
effective, but only within relatively narrow limits. Complex language features,
such as generic types and asynchronous code, were not supported. Moreover, the
entire toolchain was reliant on a fragile integration with IntelliJ, which
quickly became outdated. As as result, REM remained an academic proof-of-concept
rahter than a tool developers could realistically adopt.

This chapters details how REM has been transformed from a fragile prototype into
a robust and flexible system. We introduce support for two lanugage features
that were previously out of reach - generic code and asynchronous functions -
both of which are pervasive in real-world Rust codebases. We then describe the
design of a standalone command-line interface (CLI), eliminating dependenceis on
third-party IDEs and making REM portable across development environments.
% TODO mention either here or later that the CLI could easily be ported into the
% Language Server Protocol (LSP) framework to provide immediate IDE support.
Finally, we present a prototype Visual Studio Code (VSCode) extension that
integrates seamlessly into one of the most widely used Rust workflows. Together
these advances greatly expand the scope and usability of REM, demonstrating that
advanced refactoring can be brought to everyday developers without sacrificing
continued improvement through research. This chapter thus sets the stage by
showing how REM's capabilities now align with the realities of modern Rust programming.

% TODO here we need to mention how the extraction algorithm is ported from
% Rust-Analyzer. It has plently of custom modifications to make it work with
% REM, but a lot of the core logic is taken from there. Addtionally, the goal is
% either to implement incremental compilation (so that we don't need to fully
% build out the AST every time) or to integrate directly into Rust-Analyzer
\section{Refactoring Generics}
\label{sec:refactoring_generics}

\section{Supporting Asynchronous Code}
\label{sec:supporting_asynchronous_code}

\section{Standalone, Editor Agnostic CLI}
\label{sec:standalone_cli}

\section{Integration with IDEs}
\label{sec:integration_ides}

\section{Summary}
\label{sec:summary_2}