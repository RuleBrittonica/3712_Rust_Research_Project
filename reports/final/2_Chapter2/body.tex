\chapter{Expanding the Capabilities of REM}
\label{chap:expanding_rem}

Automated refactoring in Rust has long been constrained by the languages strict
ownership, borrowing and lifetime rules. Early prototypes of the Rusty Extraction
Maestro demonstrated that \emph{Extract Method} refactorings could be made
effective, but only within relatively narrow limits. Complex language features,
such as generic types and asynchronous code, were not supported. Moreover, the
entire toolchain was reliant on a fragile integration with IntelliJ, which
quickly became outdated. As a result, REM remained an academic proof-of-concept
rather than a tool developers could realistically adopt.

This chapter details how REM has evolved from a fragile prototype into a robust, modular, and extensible system. At its core, REM now operates through a streamlined, multi-component pipeline designed for speed, flexibility, and integration with modern Rust workflows.

The REM server orchestrates the entire process through a JSON-RPC interface, coordinating extraction, repair, and verification processes (the latter discussed in Chapter 3). The extraction tool, built directly on top of Rust Analyzer’s incremental analysis engine, performs lightning-fast semantic extraction while integrating the capabilities of REM’s original controller and borrower modules. The repairer tool, adapted from the original REM toolchain, continues to play a key role in resolving lifetimes and other semantic inconsistencies that may arise during refactoring.

Finally, a prototype Visual Studio Code (VSCode) extension connects with the REM server to deliver near-instant feedback—typically around 200 ms end-to-end (see Chapter 4). Together, these components transform REM into a practical refactoring system capable of integrating into everyday Rust development while preserving its foundation as a platform for continued research.

\section{Motivation}
\label{sec:expanding_motivation}

\section{Architecture Overview}
\label{sec:architecture_overview}

\section{Single-File Analysis}
\label{sec:single_file_analysis}

\section{Extraction Workflow}
\label{sec:extraction_workflow}

\section{Handling Advanced Language Features}
\label{sec:handling_advanced_language_features}

\section{Editor Integration}
\label{sec:editor_integration}

\section{Future Directions}
\label{sec:expanding_future_directions}