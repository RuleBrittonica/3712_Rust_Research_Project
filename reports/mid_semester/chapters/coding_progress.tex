\section{Coding Progress}

\subsection*{Modifications to REM}

The most important modification to the original toolchain has been decoupling it
from the rust compiler. Previously, REM was heavily tied to rustc, the rust
compiler. It relied compiler specific representations of files and their syntax
trees. Because of this dependency, REM required a very specific buildchain, and
could not be used outside of this build context. Additionally, when I took over
the project, changes to external libraries meant that the toolchain was
completely non-functional. Some work is still required to bring the toolchain
off of nightly rust onto stable rust.\\
Another small modification has been changing the way REM handles data.
Previously, the three main components of REM (controller, borrower and repairer)
worked by each reading and writing to a file, and then passing the file to the
next stage of the toolchain. By changing the way data is passed between the (now
5) modules of REM, a roughly 10-20\% speedup has been achieved.

\subsection*{REM-Extract: Performing inital function extraction in Rust}
% Mention HIR here, how Rust performs its type inference (and what type
% inference is in general) - \cite{DUGGAN199637}
% Also mention the extract method literature review here \cite{ExtractMethodLitReview}

\subsection*{REM-CLI: A comprehensive command line interface for REM}

The REM-CLI provides a unified interface for developers to interact with the
5 separate modules of REM. It streamlines the process of performing method
extraction by offering an intuitive command-line interface that handles
everything from code analysis to extraction. By consolidating the toolchain into
a single, easy-to-use CLI, REM removes the need for users to manually manage
multiple stages of the extraction process. It is also needed to itegrate the
functionality into a code editor.

\subsection*{REM-VSCode: A Visual Studio Code extension for REM}

The REM-VSCode extension is a proof of concept that demonstrates how the REM
tool can be easily integrated into any code editor. It provides a user-friendly interface
within the far more popular Visual Studio Code editor. Because the entire
backend of REM is now written in Rust, the extension is able to be independent
of a specific platform, and thanks the the previous work on decoupling the tool
from rustc, it no longer requires a very specific environment / buildchain to
run. It is available on the \href{https://marketplace.visualstudio.com/items?itemName=MatthewBritton.remvscode&ssr=false#overview}{VSCode Marketplace}.