\section{Theory Work Completed to Date}

\subsection*{Literature Review Progress}

A literature review is in the process of being written, and is around 60-75\%
complete. A draft will be completed by the Christmas break, and it will be
finished by the end of the summer break (although it will still be subject to
review and future revisions). It (broadly) covers the following topics, with
topics marked with * potentially being added:
\begin{itemize}[itemsep=0pt]
    \item An Overview of Extract Method Refactoring.
    \item Rust's Memory Management Model (Ownership and Borrowing), and how its'
    type system both helps and hinders refactoring tools.
    \item * Evolving Standards and Best Practices in the Rust Ecosystem.
    \item Existing Refactoring Tools and Their Limitations.
    \item Previous Work on the REM Tool by Ilya Sergey et al and Sewen Thy.
    \item Advanced Refactoring Techniques in Other Languages.
    \item Testing and Validation Strategies for Refactoring Tools
    \item * Comparative Analysis with Other Programming Languages.
    \item * Integration with Developer Tools.
\end{itemize}

\subsection*{Research into Advanced Refactoring Methods}

Recent research on advanced refactoring methods, primarily focused on Java,
highlights the importance of enhancing tools to handle the complexities of
asynchronous programming and generics. Lin et al. developed ASYNCDROID \cite{AndroidAsncRefactoring}, a tool
that refactors improperly used Android async constructs, transforming them into
more efficient forms like \verb|IntentService|. This approach
addresses memory leaks and lost results, by automating these complex refactorings. \\
Similarly, Zhang et al. proposed ReFuture \cite{AutomaticRefactoringAsyncJAVA}, a tool for automating refactoring in
Java projects using CompletableFuture, improving asynchronous code efficiency
through static analysis techniques like visitor pattern and alias
analysis. This tool demonstrated high effectiveness in large
codebases, emphasizing the need for advanced analysis to manage dependencies and
optimize task scheduling. \\
Additionally, Marticorena et al. explored the impact of generics on method
extraction refactorings in Java, noting the necessity for refactoring tools to
evolve as programming languages incorporate more sophisticated type
systems \cite{GenericRefactoringJAVA}. \\
These studies, while centered on Java, underscore the broader need for
developing refactoring methods for modern language features, particularly in
environments with strict memory and type constraints.